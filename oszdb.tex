\documentclass[12pt]{article}
\usepackage{geometry}

\usepackage[utf8]{inputenc}
\usepackage[polish]{babel}
\usepackage{polski}
\usepackage{hyperref}
\usepackage{graphicx}
\usepackage{verbatim}

\hypersetup{linkbordercolor=1 1 1}
\pagestyle{headings}

\title{Wykłady z OSZBD}

\begin{document}
\maketitle
\newpage
\tableofcontents
\newpage

\section{Autorzy}
\begin{itemize}
\item dr Robiert Marcjan -- prowadzenie wykładu
\item Michał Bugno -- \LaTeX, treść
\item Antek Piechnik -- treść
\end{itemize}

\section{Wstęp}
\subsection{Literatura}
\begin{itemize}
\item Silberschatz ``Database System Concepts``
\item Garcia-Molina, Ullman, Widom ``Database systems The Complete Book``
\item (!)Garcia-Molina, Ullman, Widom ``Implementacja Systemow Baz Danych``
\item Ramakrishan, Gecherke ``Database Managment Systems``
\item Date ``Wprowadzenie Do Baz Danych`` (uzupełniająca, raczej nie)
\item Celko ``SQL zaawansowane techniki programowania``
\end{itemize}

\subsection{Omawiane bazy danych}
\subsubsection{Komercyjne}
\begin{itemize}
\item Oracle
\item IBM INFORMIX
\item IBM DB2
\item MS SQL
\item Sybase
\end{itemize}

\subsubsection{Darmowe}
\begin{itemize}
\item PostgreSQL
\item MySQL
\end{itemize}

\subsection{Linki}
\begin{itemize}
\item www.sqlservercentral.com
\item www.databaseweekly.com
\item www.oracle.com/oramag/index.html
\item www.db2mag.com
\item www.iiug.org
\end{itemize}

\subsection{Plan}
\subsubsection{Semestr zimowy}
\begin{enumerate}
\item Architektura SZBD
\begin{itemize}
\item pamięć, procesy, dysk
\item moduły funkcjonalne SZBD, architektura klient/serwer
\end{itemize}
\item Fizyczna struktura danych
\begin{itemize}
\item struktura fizyczna zbiorów
\item metody przechowywania danych, organizacja dysków, RAID
\item fragmentacja/partycjonowanie
\end{itemize}
\item Metody dostępu do danych, indeksowanie
\begin{itemize}
\item dostęp sekwencyjny
\item indeksy, B-drzewa, hash, bitmap, siatkowe (struktury wielowymiarowe)
\end{itemize}
\item Przetwarzanie transakcyjne
\begin{itemize}
\item pojęcie transakcji
\item ACID (atomicity, consistency, isolation, durability)
\end{itemize}
\item Metody realizacji przetwarzania transakcyjnego
\begin{itemize}
\item ACD
\end{itemize}
\item Równoczesny dostęp do danych
\begin{itemize}
\item poziomy izolacji
\item metody zarządzania równoczesnym dostępem
\item blokady, wielowersyjność
\end{itemize}
\item Realizacja operacji, optymalizacja
\begin{itemize}
\item algebra relacji
\item SQL, PL/SQL i podobne
\item sposoby realizacji operacji (SELECT, JOIN itp.)
\end{itemize}
\item Replikacja danych
\begin{itemize}
\item metody replikacji danych
\item synchronizacja danych
\end{itemize}
\item Bezpieczeństwo
\begin{itemize}
\item integralność, transakcje
\item archiwizacja, odtwarzanie
\item fault tolerance (RAID, klastry, replikacja)
\end{itemize}
\item Administracja i optymalizacja wydajności serwera
\begin{itemize}
\item konfiguracja (specyficzne cechy SZBD)
\item strojenie
\end{itemize}
\item Przetwarzanie informacji przestrzennych, GIS
\begin{itemize}
\item metody, algorytmy, struktury danych
\item Oracle spatial, DB2, INFORMIX, MS SQL 2008
\end{itemize}
\item Przykłady SZBD
\begin{itemize}
\item IBM/Informix
\item IBM/DB2
\item Oracle
\item MSSQL Server
\end{itemize}
\end{enumerate}

\subsubsection{Semestr letni}
\begin{enumerate}
\item Hurtownie danych
\item Analiza danych
\item Exploracja danych
\end{enumerate}

\subsection{Laboratoria}
\begin{itemize}
\item podział na grupy na 2 semestry
\item Prowadzący
\begin{itemize}
\item Anna Zygmunt (pon. 11.oo - 12.3o, pt. 15.3o - 17.oo)
\item Robert Marcjan (pon. 9.3o - 11.oo)
\item Leszek Siwik (pt. 15.3o - 17.oo)
\end{itemize}
\end{itemize}

\section{Architektura SZBD}
%* modele danych
%* szbd
%* architektury systemow baz danych
%* budowa wewnetrzna

\begin{description}
\item[Model danych] zestaw pojęć używanych do opisu danych
\end{description}

Modele danych:
\begin{itemize}
\item związków encji
\item sieciowy
\item hierarchiczny
\item relacyjny (mocna teoria (algebra relacji), udany, prosty)
\item obiektowy (nie tak wydajny jak relacyjny)
\item obiektowo-relacyjny
\end{itemize}

\subsection{Relacyjny model danych}
\subsubsection{Cechy}

\begin{itemize}
\item relacje, krotki, atrybuty
\item tabele, wiersze, kolumny
\item algebra relacji (prosta, przejrzysta)
\begin{itemize}
\item wejście: relacja1, relacja2 ..., wyjście: nowa relacja
\item operacje: selekcja, projekcja, złączenie (join), suma, różnica, przecięcie etc.
\end{itemize}
\item język SQL
\end{itemize}

\subsubsection{Niezależność}
\begin{itemize}
\item możliwość modyfikacji definicji (schematu) na danym poziomie bez konieczności zmian na wyższym poziomie
   (poziom fizyczny, logiczny, prezentacji)
\item aplikacje są niezależne od tego jak dane są strukturalizowane i przechowywane
\end{itemize}

\subsubsection{Architektury}
\begin{itemize}
\item  mainframe
\item  file sharing, ISAM (Index Sequential Access Method) (na plikach)
\item  client-server (klient nie ``dotyka`` serwera)
\item  architektury wielowarstwowe (3 warstwowe)
\item  architektury rozproszone
\begin{itemize}
\item  rozproszenie danych
\item powielanie (replikacja) danych
\item transakcje rozproszone
\end{itemize}
\item  architektury równoległe (w tym widzi się skok wydajności, dotychczasowe rozwiązania są już
   bardzo dobrze zoptymalizowane)
\end{itemize}

\subsubsection{Client--server}
\begin{itemize}
\item operacje ``klienta``
\item operacje ``serwera``
\item SQL
\item stored procedures -- każdy ma własny język, nie ma zgodności (zmienne, pętle, nie ma struktur danych)
\item SQL + możliwość wykonania programu na serwerze (SUM(kolumna) vs SELECT * a potem sumuj, programy w Java, C)
\end{itemize}

\subsubsection{Dlaczego SZDB?}
\begin{itemize}
\item redundancja danych
\item spójność danych (dlaczego nie tylko w aplikacji? bo z reguły SZDB żyje DŁUŻEJ niż aplikacja)
\item równoczesny dostęp do danych (nie trzeba przejmować się równoczesny dostępem)
\item transakcje (atomicity, conistency, isolation, durability)
\item bezpieczeństwo danych
\begin{itemize}
\item ochrona przed niepowołanym dostępem (bardzo dokładna kontrola np. konkretne kolumny)
\item odporność na awarię
\end{itemize}
\end{itemize}
Przetwarzanie powinno być \emph{efektywne}.

\subsection{Pamięć, dyski}
W SZDB dane przechowywane są na dyskach. Ma to znaczenie przy ich projektowaniu.
\begin{comment}
 * READ -- transfer z dysku do RAM
 * WRITE -- zapis z RAM na dysk
 * te operacje są kosztowne w porównaniu do operacji w pamięci

Server: pamięć, procesy komunikujące i dysk

Komponenty tworzące SZDB:
 * dysk
  * struktury przechowujące dane
  * struktury przechowujące opis danych
  * struktury zapewniające przetwarzanie transakcyjne
  * struktury zapewniające bezpieczeństwo
 * procesy
  * operacje związane z funkcjonowaniem serwera
  * operacje wykonywane ``na zlecenie`` aplikacji klienckich
 * pamięć
  * pula buforów
  * struktury kontrolne

Typowy system ma dużo więcej danych na dysku niż może przechowywać w pamięci.

 * RAID
 * backup danych, backup logów (raz na jakiś czas zapisujemy dane, ciągle backupujemy logi)
  * w przypadku awarii restore danych a potem wykonujemy logi od ostatniego dobrego backupu

 * replikacja
  * zabezpieczenie przed utratą wszystkiego (serwer wykonuje log i przesyła go gdzieś indziej, do innego serwera)
  * przybliżenie użytkownikowi danych (coraz mniejsza rola tej funkcji, przepustowość sieci rośnie)

Struktury dyskowe
 * czas dostępu do dysku jest bardzo wolny
 * operujemy na dużych zbiorach danych
 * powinny zapewniać wydajność i bezpieczeństwo

Storage-device hierarchy:
  cache (najszybszy) <=> main memory <=> flash memory <=> magnetic disk <=> optical disk <=> magnetic tapes (najwolniejszy)

Dysk:
 * głowica
 * talerz
 * ścieżka
 * cylinder
 * blok
 * sektor

time = seek time (head)(longest?) + rotation delay + transfer time

Dostęp:
 * random (1o-krotni wolniejsze)
 * następny blok
 * sekwencyjny

Przykład:
transfer rate = 4ooo KB
avg seek time = o.o1 s
page          = 2KB

* random (1 strona): (2 KB / (o.o1 s + (2 KB / ( 4ooo KB/s )))) = 19o KB/s
* sekwencyjny (3o stron): (60 KB / (o.o1 s + (60 KB / ( 4ooo KB/s )))) = 24oo KB/s

 * odpowiednia organizacja przestrzeni dyskowej
 * szeregowanie operacji we/wy (sterowanie ruchem głowicy)
 * bufor dla zapisu (np. w pamięci RAM)
 * bufor na dysku (zapis sekwencyjny do bufora a potem porozdzielanie w odpowiednie miejsca)

RAID:
 * Redundant Array of Inexpensive Disks
 * Redundant Array of Independent Disks
 * poprawa wydajności - zrównoleglenie operacji we/wy
 * poprawa bezpieczeństwa - nadmiarowość

MTTF - Mean Time To Failure - średni czas spodziewanej bezawaryjnej pracy dysku
Typowy MTTF - kilka-kilkadziesiąt lat
Prawdopodobieństwo awarii jest stosunkowo niskie

Poprawa wydajności:
 * bez nadmiarowości
  * 1 dysk: ~1oo.oooh ~ 11 lat
  * 100 dysków: ~1oooh ~41dni
 * mirroring
  * średni czas 1oo.oooh ^ 2 / 10 = ~1oo.ooo lat
  * ale prawdopodobieństwo awarii rośnie mocno z wiekiem
  * rzeczywisty czas rzędu kilkudziesięciu lat
 * Bit-level striping
  * np macierz 8 dysków, i-ty bit umieszczony na i-tym dysku
  * 8x szybszy dostęp
 * Block-level striping

RAID0 -- wysoka wydajność, utrata danych nie może być problemem
RAID1 -- wydajny ale kosztowny
RAID3 -- szybki transfer danych
RAID5 -- lepszy niż RAID3 dla dostępu typu ``random``
RAID > 5 -- raczej niestetosowany

Jeśli stać nas na RAID 0+1 to jest najlepsze rozwiązanie, jeśli nie to RAID5

Inne sposoby poprawy wydajności:
 * mirroring na poziomie SZDB i systemu
 * sterowanie położeniem danych na dysku (bliżej środka)
 * fragmentacja/grupowanie danych
  * klastry
  * fragmentacja tabel
  * separowanie danych na poziomie fizycznym

W jaki sposób przechowywać dane:
 * plik, rekord, blok
 * tabela, wierz, indeks, baza
 * relacja między poziomem logicznym a fizycznym
  * plik składa się z bloków (a nie pojedynczych bitów [odczyt])
  * format rekordu: stały lub zmienny

Schemat rekordu (stały):
 * zwiera
  * liczba pól
  * typ każdego pola
  * kolejność pól w rekordzie
 * definiuje rekord

 * Zalety
  * umożliwia łatwy dostęp

 * Wady
  * prostota implementacji
  * tylko jeden typ rekordy w danym pliku
  * dopisywanie na koniec
  * usuwanie rekordów
   * flaga `deleted`
   * lista usuniętych rekordów (dodawanie może ją wykrzoystywać)
   * jako że więcej INSERT niż DELETE to nie jest problem
  * jeśli rozmiar bloku nie jest wielokrotnością rozmiaru rekordu to niektóre
    rekordy będą wymagały dwóch odczytów

Schemat rekordu (zmienny):
 * każdy rekord zawiera opis formatu (self describing)
 * Zalety
  * rzadkie rekordy
  * formaty zmienne w czasie
  * rekordy różnych typów w jednym pliku
 * Wady
  * marnuje ilość miejsca (więcej miejsca == więcej czasu do przetwarzania danych)
  * trudniejszy w implementacji

Oszczędność miejsca:
 * sposoby przechowywania NULLi
 * wskaźniki
 * rezerwowanie miejsca dla rekordów (np rezerwowanie 255 dla VARCHAR _zawsze_)

Format sloted page:
 * plik składa się ze stron które odpowiadają blokom na dysku (2kB (Oracle), 4kB, 8kB)
 * nagłówek (timestamps etc.)
 * tablica slotów (wskaźnik do początku wolnego miejsca, wskaźnik -- rozmiar, wskaźnik -- rozmiar ...)
 * elastyczny
 * powszechnie stosowany w SZBD
 * w praktyce:
  * jedna strona == jedna tabela
  * sekwencyjne przetwarzanie danych z tabeli
 * wiele tabel w jednym pliku
 * jedna tabela w wielu plikach
 * reguła: jeżeli wiersz mieści się na stronie, to jest zawsze na jednej stronie
  * więcej zajętego miejsca
  * efektywniejsze przetwarzanie danych

Sposoby reprezentacji:
1. Każdej tabeli odpowiada plik.
 * mniejsze SZBD
2. Rezerwujemy obszar dysku
 * większe SZBD

Przetrzeń dyskowa SZBD:
 * Oracle -- tablespace
 * Informix -- Dbspace
 * MSSQL, Sybase -- file group

  Informix
   * database, dbspace, chunk
   * przestrzeń dyskowa z jednego lub więcej chunków
   * chunk to jeden plik lub obszar dyskowy
   * przestrzeń dyskowa podzidzelona na jednostki logiczne (dbspace), każda
     składa się z jednego lub więcej plików
   * na serwerze może być wiele BD (w określonych Dbspaceach)

  MSSQL
   * database, filegroup, datafile
   * na serwerze wiele baz danych
   * dla każdej bazy danych mamy przyzielaną przestrzeń dyskową
   * dla każdej bazy można stworzyć kilka grup plików (file group)

  ORACLE
   * Tablespace, datafile
   * przestrzeń dyskowa podzielona na tablespace'y
   * tablespece to jeden lub więcej plików

Tabela
 * W momencie tworzenia serwer przydziela jej przestrzeń dyskową składającą
   się z obszaru stron na dysku nazywanego zakresem (extent)
 * extent: ciągly obszar stron na dysku
 * rozmiar zwykle kilka lub kilkanaście stron (można nim sterować)
 * dopisywanie wierszy do tabeli powoduje wypełnianie stron w ramach extentu
 * jeżeli brakuje miejsca to przydzielany jest kolejny extent

Rozmiar extentu może być zmieniony (od danego momentu extenty większe). Można
też automatycznie zwiększać rozmiar extentu (np. co 16 extent podwajamy)
Extenty nie są zwalniane automatycznie (wyjątek: tabele tymczasowe) -- dlatego
że zakładamy, że z reguły tabele się rozrastają a nie kurczą.

Oracle (PCTFREE, PCTUSED)



WYKLAD - 03 / 11 / 2008



Tabela - podsumowanie (zeszly wyklad)
	* Dane tabel przechowywane sa w stronach plikow z danymi
	* Wiersz tabeli odpowiada slotowi na stronie
	* Mozna sterowac fizycznym polozeniem danych na dyskach przez umieszczenie tabeli w logicznej jednostce serwer
	* Przestrzen przydzielana jest w extentach
	* Extent'y nie sa zwalniane automatycznie
	* Dane tabeli moga znajdowac sie w kilku plikach

Cluster
  	* Przechowywanie dwoch (lub wiecej) tabel lacznie na poziomie fizycznym (razem)
		- jesli najczestszym sposobem ich przegladania jest join
		- mozna stworzyc strukture typu cluster
	* ORACLE:
		CREATE TABLE dept (bla bla)
		CLUSTER emp_dept(deptno);
	
Fragmentacja tabel
	* dystrybujcja dany z jednej tabeli pomiedzy wiele dbspace'ow)
 	* INFORMIX:
	FRAGMENT BY ROUND ROBIN (bez konkretnego pomyslu)
		IN dbspace1, dbspace2, dbspace3
	FRAGMENT BY EXPRESSION
		employee_num <= 2500 IN dbspace1,
		employee_num > 2500 IN dbspace2,
		etc.	
	
	* ORACLE:
	PARTITION BY RANGE (week_no)
		(PARTITION sales1 VALUES LESS THAN (4) TABLESPACE ts1)

Podsumowanie
	* Dazymy do tego zeby tabele byly przechowywane w ciaglych obszarach na dysku
		- w ogolnym przypadku - trudne
		- pewnym rozwiazaniem jest pamietanie danych w ciaglych extentach
		- warto czasem zreorganizowac strukture extentow aby uzyskac ciagly obszar
		
	* Czasami sens ma partycjonowanie/fragmentacja danych
		- zrownoleglanie operacji we/wy
		- zysk w przypadku wielu malych transakcji
		- zysk w przypadku dlugich operacji
		
	* W pewnych przypadkach sens ma tworzenie struktur typu cluster
		- gdy wiekszosc operacji dla tych tabel operacje pobierania danych to JOIN
		- tabele nie sa przegladane samodzielnie ('podatek')
	
Indeksy
	* Indeksy plaskie
		- uporzadkowane zbiory indeksujace
	* Indeksy bazujace na B-drzewach
	* Indeksy - struktury typu hash
	* Indeksowanie po kilku atrybutach
	* Pojecie ogolne
		- Moze to byc plik plaski, struktura drzewiasta, hash
	* Moze byc uzywany na dwa sposoby
		- W celu uzyskania dostepu o charakterze bezposrednim
		- W celu uzyskania dostepu sekwencyjnego (dostep zgodny z porzadkiem wyznaczonym przez przez indeks)
	* Indeksowanie jest uzywane zeby przyspieszyc dostep
	* Klucz (search key) - atrybut/zbior atrybiutow wg ktorego wyszukujemy
	* Zbior indeksujacy - rekordy postacji search-key<-->pointer
	* Indeks jest mniejszy od zbioru ktory podlega indeksowaniu
	* Dwa podstawowe typy indeksow
		* uporzadkowane - klucze sa uporzadkowane
		* hash - klucze sa rozmieszczone rownomiernie (zgodnie z funkcja haszujaca)
		
Miary efektywnosci indeksow
	* Dostep do danych
		- rekordy o okreslonej wartosci atrybutu (rekordy ktory spelniaja warunek)
		- rekordy ktore nie spelniaja warunku dla okreslonego atrybutu
	* Czas dostepu do danych
	* Czas potrzebny na dopisanie rekordu
	* Czas potrzebny na usuniecie rekordu
	* Narzut wynikajacy z przestrzeni zajmowanej przez indeks
	
Indeksy primary/secondary
	* Primary
		- uporzadzkowanie wg indeksu zgodne z fizycznym uporzadkowaniem danych
		- czasami nazywane grupujacymi (clustering index)
	* Secondary
		- uporzadkowanie nie jest zgodne fizycznym uporzadkowaniem
		- nazywane non clustering index

Indeksy Primary
	* Indeks gesty (dense index) - rekord indeksu dla kazdej wartosci klucza
	* Indeks rzadki (sparse index)
	* Indeksy wielopoziomowe (plaski indeks do plaskiego indeksu)
	* Duplikaty (reprezentowanie tylko raz wartosci klucza w pliku - optymalnego rozwiazanie, wciaz indeks gesty)
	
Indeksy Secondary
	* Indeksy rzadkie - nie maja sensu
	* Duplikaty (buckets)

Indeksy - Uwagi
	* Generalnie indeks przyspiesza dostep do danych
	* Dodatkowy koszt zwiazany z modyfikacja danych
	* Jesli indeks jest duzy to spada wydajnosc dostepu do danych (dla plaskich plikow indeksujacych)
	* Dostep sekwencyjny 
		- dostep sekwencyjny przy uzyciu indeksu primary - efektywny
		- dostep sekwencyjny przy uzyciu indeksu secondary - o wiele bardziej kosztowny
	* Indeksy rzadkie sa mniejsze
		- ale trzeba dodatkowo realizowac wyszukiwanie na poziomie danych
	* W przypakdu indeksow gestych niektore operacje moga byc zrealizowane przy pomocy samego indeksu
		- w przypadku rzadkich w zasadzie tez, ale jest zdecydowanie mniej mozliwosci
		
Indeksy wielopoziomowe - B-drzewa
	* B- balanced - zrownowazone
	* Wezel
		K_1, P_1, K_2, P_2, ...., K_n, P_n
		- K - klucz
		- P - pointer
		- n - rozmiar
		- klucze sa uporzadkowane
		

B+ drzewa (wypelnienie wezlow)
  * Glebokosc B+ drzewa
    - K- liczba kluczy
    - Log_[n/2](K)
    - np. K = 1 000 000
          n = 100
          glebokosc = 4

  * B drzewa - najczestszy sposob organizacji
    - wydajny jesli chodzi o dostep "random"
    - wydany jesli chodzi o dostep sekwencyjny

Hash
  * Funkcja haszujaca
    - Key - > h(key)
  * Struktury typu hash
    - Statyczne
    - Dynamiczne (linear hashing, extensible hashing)
  * Podsumowanie
    - Struktury typu hash daja lepsze efekty jesli chodzi o dostep bezposredni
    - Problemy przy zapytaniach typu : zakres wartosci

SQL - indeksy

  CREATE INDEX <index-name> ON <relation-name> (<attribute-list>)

  CREATE INDEX b-index ON branch(branch-name)

  CREATE UNIQUE INDEX b-index ON branch(branch-name)

  DROP INDEX <index-name>

  FILL FACTOR <procent>

  FILL FACTOR 10%

.
WYKLAD 17/11/2008

Zagadnienia do rozważenia:
 * możliwość awarii systemu podczas transakcji
 * równoczesny dostęp do danych, równoczesne wykonywanie transakcji

Własności transakcji
 * consistency
    read(A)
    A = A - 50
    write(A)
    read(B)
    B = B + 50
    write(B)
    A + B = const
    na przykład awaria między 3 i 4 powoduje, że baza powróci do stanu pierwotnego
 * trwałość
    zmiany mają charakter trwały
 * izolacja
    transakcje są wykonywane w całowitej izolacji (dla innych transakcji nie są widoczne
    do czasu zatwierdzenia

Stany transakcji:

 active -> partially commited -> commited
 active -> failed
 partially commited -> failed -> aborted

 Gdy nastąpi awaria, to tranakcja zostaje w stanie dowolnym (!)
 Po restarcie SZBD przegląda niedokończone transakcje i doprowadza je do końca

W jaki sposób zapewnić transakcyjność?

 * shadow database: kopiujemy bazę, i po udanej transakcji usuwamy starą bazę i wskaźnik wskazuje na nową
   (izolacja: po kolei)
   - prosta implementacja
   - nieefektywna (kopiowanie _całej_ bazy danych?!)
   - krótkie transakcje czekają na jedną długą

 * algorytmy zapewniające niepodzielność i trwałość
   Metody pozwalające odtworzyć stan bazy w przypadku awarii, powszechnie stosujemy system logów.
   - awarie transakcji:
    * użytkownik - rollback (jawnie)
    * serwer - na przykład naruszenie warunku integralnościowego
   - awarie aplikacji
    * awaria połączenia itp
   - awarie na poziomie serwera baz danych
    * system crash (utrata danych w pamięci)
    * uszkodzenie dysku (utrata danych na dysku)

   1. Akcje podejmowanie podczas normalnego wykonywania się transkcji
   2. Akcje podejmowane po awarii w celu odtworzenia stanu bazy

   INPUT(X) -- blok z dysku zawierający element X jest kopiowany do pamięci
   READ(X, t) -- jeżeli w pamięci nie ma bloku X, to INPUT(X), następnie t := x
   WRITE(X, t) -- jeżeli bloku zwierającego X nie ma w pamięci to INPUT(X), następnie X := t
   OUTPUT(X) -- zawartość bufora kopiowania na dysk

              | t  | MA | MB | DA | DB | LOG
   READ(A, t) | 8  | 8  |    | 8  | 8  | <START...>
   t := t * 2 | 16 | 8  |    | 8  | 8  |
   WRITE(A, t)| 16 | 16 |    | 8  | 8  |
   READ(B, t) | 8  | 16 | 8  | 8  | 8  |
   t := t * 2 | 16 | 16 | 8  | 8  | 8  |
   WRITE(B, t)| 16 | 16 | 16 | 8  | 8  |
   OUTPUT(A)  | 16 | 16 | 16 | 16 | 8  |
   OUTPUT(B)  | 16 | 16 | 16 | 16 |16  |

  LOG:
  <START T> -- rozpoczęcie transakcji T
  <T, X, v> -- T -- transakcja, X -- el. danych, v -- wartość przed modyfikacją
  <COMMIT T> -- zakończenie transakcji
  Log może służīć do operacji undo (wycofania transakcji)

  Rozwiązanie logów:
   * rekord logu dla każdej operacji
   * rekord logu musi osiągnąć dysk zanim zmodyfikowany bufor zostanie zapisany
     na dysk (WAL -- write ahead logging)
   * przed commitem wszystkie operacje muszą być zapisane na dysk

  W czasie awarii:
   * <ABORT T> do logu
   * jeśli transakcja zapisuje informacje na dysk dopiero po zakończeniu
     wszystkich operacji to nie trzeba robić odtwarzania w przypadku błędu
     (wystarczy abort do logu)
   * transakcje niekompletne: jeśli jest <START T> a nie ma <COMMIT T> ani <ABORT T>
   * podczas procesu odtwarzania awaria nie jest problemem: po prostu powtarzamy proces
     odtwarzania

  Operacja odtwarzania wymaga przeglądnięcia całego logu
  Rozwiązanie: co jakiś czas CHECKPOINT, czyli:
    * zatrzymaj ``nowe transakcje``
    * poczekaj aż bieżące transakcje się skończą (COMMIT lub ABORT)
    * zapisz bufory na dysk
    * zapisz informacje o CHECKPOINTcie do logu (<CHECKPOINT>)
  Wtedy możemy zacząc odwarzanie od CHECKPOINTu. Typowy system ma CHECKPOINT co kilka minut.

  Minusy algorytmu CHECKPOINT:
    * żadnych nowych transakcji w tym czasie
    * czekanie na koniec transakcji

  Rozwiązanie:
  NONQUISCENT CHECKPOINT
   * <START CHECKPOINT(AT1, AT2...)> (ATN - aktywana transakcja N)
   * nie wstrzymujemy nowych transakcji
   * po zakończeniu transakcji z listy <END CHECKPOINT>

   Awaria w trakcie CHECKPOINTu, trzeba:
    * odtworzyć transakcje, które rozpoczęły się po <START CHECKPOINT>
    * wszystkie transakcje z listy przy <START CHECKPOINT>

   Awaria po <END CHECKPOINT>
    * tylko transakcje po <START CHECKPOINT>

  UNDO:
   * pewne transakcje które wyglądają jakby były zaitwerdzone mogą zostać wycofane
   * COMMIT dopiero po zapisaniu wszystkich danych na dysk
   * problem w przypadku odtwarzania backupu

  REDO:
   * wymaga modyfikowania ...

  Redo/undo logging:
   * rekord logu zawiera starą i nową wartość: <T, X, v, w>
   * rekord logu dla każdej operacji
   * zapisz log na dysk zanim bufor X zostanie zapisany na dysk
   * zapisz log po każdym <COMMIT T>

   * rolling back:
    - konstruuj listę zatwierdzonych transakcji -- S
    - jeśli transakcja nie jest na liście S wykonaj UNDO
   * rolling forward
    - wykonaj operację REDO dla transakcji z listy S

    ..........dbdump.................lastneededundo................checkpoint
    not needed             not needed                    not needed
     for media         for undo after                for redo after
      recovery         system failure                system failure
\end{comment}
\end{document}
